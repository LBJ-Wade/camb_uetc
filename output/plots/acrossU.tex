%JZ had it last
%\documentclass[prd,aps,tightenlines]{revtex4}
%\documentclass[twocolumn,prd,aps,showpacs,superscriptaddress]{revtex4}
\documentclass[twocolumn,aps,prd,showpacs]{revtex4}
%\usepackage{natbib}
\usepackage{amsmath}
%\usepackage{verbatim}
%\usepackage{dcolumn}
\usepackage{graphicx}
\usepackage[usenames]{color}
%\usepackage{psfrag}

% ---------------- May need to comment this out ----------------
%\usepackage{epstopdf}
%\usepackage{amsbsy,amssymb,amsmath}
%\usepackage{rotating}

\usepackage{psfrag}
\usepackage[ps2pdf,colorlinks,bookmarks]{hyperref}
%\usepackage{hyperref}
\definecolor{linkblue}{rgb}{0,0,0.8}
\definecolor{linkgreen}{rgb}{0,0.5,0}
\hypersetup{pdfpagemode=None, pdfstartview=FitH, linkcolor=linkblue, %
             citecolor=linkgreen, urlcolor=linkblue}
\newcommand{\note}[1]{{\color{red}[#1]}}
%\newcommand{\note}[1]{{[#1]}}

\bibliographystyle{doiarxiv}
%\bibliographystyle{prsty}

\def\nlb{\nolinebreak}
\newcommand{\ud}[2]{^{#1}_{\phantom{#1}#2}}
\newcommand{\du}[2]{_{#1}^{\phantom{#1}#2}}
\newcommand{\udu}[3]{^{#1\phantom{#2}#3}_{\phantom{#1}#2}}
\newcommand{\udud}[4]{^{#1\phantom{#2}#3}_{\phantom{#1}#2\phantom{#3}#4}}
\newcommand{\uds}[2]{^{#1{\rm(s)}}_{\phantom{#1}#2}}
\newcommand{\udv}[2]{^{#1{\rm(v)}}_{\phantom{#1}#2}}
\newcommand{\udt}[2]{^{#1{\rm(t)}}_{\phantom{#1}#2}}
\def\beq{\begin{equation}}
\def\eeq{\end{equation}}
\def\bea{\setlength\arraycolsep{1.4pt}\begin{eqnarray}}
\def\eea{\end{eqnarray}}
\def\bit{\begin{itemize}}
\def\eit{\end{itemize}}
\def\dsp{\displaystyle}
\def\etal{{\it et~al.}}
\def\ie{{\it i.e.}}
\def\eg{{\it e.g.}}
\def\vs{{\it vs.}}
\def\eq{Eq.~}
\def\eqs{Eqs.~}
\def\fig{Fig.~}
\def\figs{Figs.~}
\def\pd{\partial}
\def\lap{\nabla^2}
\def\lapp{\fr{1}{a^2}\nabla^2}
\def\sdot{^{\displaystyle\cdot}}
\def\ld{\left}
\def\rd{\right}
\def\ra{\rightarrow}
\def\tl{\tilde}
\def\wtl{\widetilde}
\def\ph{\phantom}
\def\fr{\frac}
\def\oo{\frac{1}}
\def\half{\frac{1}{2}}
\def\const{{\rm const}}
\def\gam{\gamma}
\def\gamd{\dot{\gamma}}
\def\del{\delta}
\def\kap{\kappa}
\def\lam{\lambda}
\def\Lam{\Lambda}
\def\vphi{\varphi}
\def\vphid{\dot{\varphi}}
\def\vphidd{\ddot{\varphi}}
\def\rhod{\dot{\rho}}
\def\sig{\sigma}
\def\Sig{\Sigma}
\def\om{\omega}
\def\ad{\dot{a}}
\def\add{\ddot{a}}
\def\aoan{\frac{a}{a_0}}
\def\hd{\dot{H}}
\def\gmn{g_{\mu\nu}}
\def\tmn{T_{\mu\nu}}
\def\lpl{l_{\rm P}}
\def\mpl{m_{\rm P}}
\def\rg{\sqrt{g}}
\def\Rs{{}^{(3)}\!R}
\def\liexi{{\cal L}_\xi}
\def\h{{\cal H}}
\def\K{{\cal K}}
\def\L{{\cal L}}
\def\O{{\cal O}}
\def\p{{\cal P}}
\def\phidot{\dot{\varphi}}
\def\delphi{\delta\varphi}
\def\delphid{\delta\dot{\varphi}}
\def\bra{\langle}
\def\ket{\rangle}
\def\sh{super-Hubble}
\def\ds{de Sitter}

\def\ehat{\boldsymbol{\hat{e}}}
\def\khat{\boldsymbol{\hat{k}}}
\def\alm{a_{\ell m}}
\def\Ylm{Y_{\ell m}}
\def\Cl{C_\ell}
\def\bk{{\boldsymbol{k}}}
\def\bx{{\boldsymbol{x}}}
\def\R{{\cal R}}
\def\PR{{\cal P_{\cal R}}}
\def\rls{r_{\rm LS}}
\def\jl{j_\ell}

%%Table stuff
%\newbox\tablebox    \newdimen\tablewidth
%\def\leaderfil{\leaders\hbox to 5pt{\hss.\hss}\hfil}
%\def\endtablenow{\tablewidth=\columnwidth
%    $$\hss\copy\tablebox\hss$$
%    \vskip-\lastskip\vskip -2pt}
%\def\endtablewide{\tablewidth=\textwidth
%    $$\hss\copy\tablebox\hss$$
%    \vskip-\lastskip\vskip -2pt}
%\def\tablenote#1 #2\par{\begingroup \parindent=0.8em
%    \abovedisplayshortskip=0pt\belowdisplayshortskip=0pt
%    \noindent
%    $$\hss\vbox{\hsize\tablewidth \hangindent=\parindent \hangafter=1 \noindent
%    \hbox to \parindent{\sup{\rm #1}\hss}\strut#2\strut\par}\hss$$
%    \endgroup}
%\def\doubleline{\vskip 3pt\hrule \vskip 1.5pt \hrule \vskip 5pt}


\begin{document}

\title{Across the Universe: constraining tilts in cosmological
parameters}
\author{Joel Hutchinson} \email{jhutchin@ualberta.ca}
\affiliation{Department of Physics \& Astronomy\\
University of British Columbia, Vancouver, BC, V6T 1Z1  Canada}

\author{Adam Moss} \email{adammoss@phas.ubc.ca}
\affiliation{Department of Physics \& Astronomy\\
University of British Columbia, Vancouver, BC, V6T 1Z1  Canada}

\author{Douglas Scott} \email{dscott@phas.ubc.ca}
\affiliation{Department of Physics \& Astronomy\\
University of British Columbia, Vancouver, BC, V6T 1Z1  Canada}

\author{James P. Zibin} \email{zibin@phas.ubc.ca}
\affiliation{Department of Physics \& Astronomy\\
University of British Columbia, Vancouver, BC, V6T 1Z1  Canada}

\date{\today}



\begin{abstract}
\note{To be rewritten}  Large-scale variations of
physical constants or cosmological parameters could manifest 
themselves as gradients across our Hubble volume, leading to correlations
between adjacent multipoles of the cosmic microwave background anisotropies.  
We perform a generic test for these correlations in {\it WMAP\/} data and note
that the multipole dependence of future estimates of the dipole
modulation could enable us to determine which parameter is varying.
For example, using a Fisher matrix analysis, we find that the
{\em Planck\/} satellite could detect a gradient
in the fine structure constant as low as
$\Delta\alpha/\alpha=2.5\times10^{-4}$ across our observable volume.
We also extend the statistical approach to distinguish the case when we
know the direction in advance from the case where a dipole modulation is
sought in any direction.
\end{abstract}
\pacs{06.20.Jr, 98.70.Vc, 98.80.Cq, 98.80.Es, 98.80.Jk}

\maketitle





%----------------- INTRODUCTION -----------------------
\section{Introduction}
\note{To be completely rewritten}  

The parameters which describe the standard model of cosmology
(see e.g.~\cite{scott06}) are usually considered to be constants,
i.e.\ it is assumed that their values are homogeneous in
space.  However, there are several situations in which we might consider
spatial variations in these parameters.  Firstly, the structure within our
observable Universe is a realization of an underlying power spectrum.  This
means that there is some stochasticity, even on the scale of our Hubble patch,
leading to a monopole fluctuation in the temperature of the cosmic microwave
background (CMB)~\cite{Zibin:2008fe} and other parameters.  Secondly,
there could be some physical process operating on large scales which makes
some parameters vary from place to place---two obvious examples are
inhomogeneous big bang
nucleosynthesis (e.g.~\cite{Harrison}) and anisotropic inflation
\cite{Gordon:2005ai,Ackerman:2007nb,Pullen:2007tu,Gumrukcuoglu:2007bx}.
The third type of possibility is to imagine that the laws of physics
themselves have a dependence on position, for example that the vacuum
density $\Lambda$ or fine structure constant $\alpha$ might be different in
different volumes.

In the first scenario, the variations are expected to be small, and are
part of the conventional cosmological picture, being contained within what
is usually referred to as ``cosmic variance''.  But in the second and
third scenarios, the putative physical mechanism is of unknown strength and
scale.  Hence to test for such ideas we need to keep an open mind and
develop generic approaches for searching for measurable effects.
If cosmological parameters (or physical laws) vary on very large scales,
then we would expect that the most prominent observable effect would be a
gradient across our Hubble patch.

In Moss, Scott, Zibin, and Battye~\cite{Moss:2010} (MSZB) we discussed a 
general method for
searching for such gradients.  This is based on the idea that a gradient
is equivalent to a dipole modulation of the CMB sky, which in a spherical
harmonic expansion results in correlations between neighbouring multipoles.
This idea had been explored earlier by several groups for the specific case
of modulation of the primordial amplitude of scalar perturbations $A_{\rm s}$
(see~\cite{Carroll2010} and references therein)
motivated by the apparent hemispheric asymmetry of the CMB sky
\cite{Eriksen:2003db,Eriksen:2007pc}.
The mathematics is also similar to that for the case of aberration of the
CMB sky, as described in several recent papers
\cite{Kosowsky:2010jm,Amendola:2010ty,Chluba:2011}.
However, the appealing
aspect of the idea is that it is quite generic, and can be used to test
for a gradient in {\it any\/} parameter.

In this paper we will develop this idea further.
We begin in the next section by
briefly describing the method used to measure correlations
on the CMB sky caused by dipole modulation. For some parameters (e.g.\ $\alpha$)
this does not result in the generation of an actual
$\ell=1$ mode. However, it is important to note that even though 
other parameters (such as $A_{\rm s}$) may generate such a temperature dipole, 
this is 
a non-linear effect, and so the size of such a mode is still very small compared 
with our motion-induced dipole. Hence there is typically no strong
constraint on gradients from the measured amplitude of the dipole itself---the 
limits come from measuring the correlations between
$\ell$ and $\ell\pm1$.  After describing the method we then show some
simulated results.  Our detailed formalism is described in Sec.~\ref{sec:cov}.
In Sec.~\ref{sec:fisher} we
show how one can use a Fisher matrix approach to estimate how well future CMB
experiments could constrain a tilt in parameters.  We also distinguish the
case where there might be more than one gradient, in different directions, and show
that this case requires only a simple modification to the case of a single gradient 
across our sky.
In Sec.~\ref{sec:wmap} we examine Wilkinson Microwave Anisotropy Probe 
({\it WMAP\/}) data, finding some evidence of a dipole modulation.
%[LAST SENTENCE TO FINISH INTRO?]

It is clear that the expectation value 
of the temperature is $0$ at both poles, but the variance at each is 
different. Thus any temperature dipole is at most a second moment effect and
will be much smaller than the dipole caused by our
motion, which has $|a_{1m}|^2\simeq10^4 |a_{\ell m}|^2$ for $\ell>1$.
It is also important to note that gradients in some parameters produce no
dipole at all.  Taking $\alpha$ as an example,
the imprint on the CMB arises entirely from a 
modification of recombination physics.  This gives an epoch of last 
scattering which is anisotropic, but this does not produce a temperature
dipole.  An increase in $\alpha$ increases the recombination rate, causing
recombination to occur at \emph{both} a higher temperature and redshift
(see MSZB), so that the observed temperature in that direction is
unchanged.

Evidence for a dipolar power asymmetry (i.e.\ the case $X = A_{\rm s}$)
in the {\em WMAP\/} 
data was noticed soon after the initial data release (see 
e.g.~\cite{Eriksen:2003db,Prunet:2004zy}).  The claimed significance of
detection has mildly increased with subsequent data releases.
Recently, two groups have studied the scale-dependence of the signal by 
limiting the likelihood to some $\ell_{\rm max}$
\cite{Hoftuft:2009rq,Hanson:2009gu}, for example the first group used an
exact Bayesian analysis, truncating at $\ell_{\rm max} = 64$ and found
$\Delta A_{\rm s}/A_{\rm s} = 0.072 \pm 0.022$ 
pointing towards $(l,b)=(224^{\circ},-22^{\circ})$, with an uncertainty
of $\pm 24^{\circ}$. 

The result of Ref.~\cite{Hoftuft:2009rq} is the detection of hemispherical
asymmetry at $3.3\sigma$ significance, but they also found hints of departures
from the $\ell$-dependence expected from modulating $X = A_{\rm s}$.  In
Ref.~\cite{Hanson:2009gu} this was extended to higher $\ell$ using an
approximate quadratic likelihood estimator, finding consistent results, albeit
at lower significance.   Note, however, that the statistical significance of
both of these results has been criticized in Ref.~\cite{Bennett:2010jb},
due to the {\it a posteriori} choice of $\ell_{\rm max}$.




%-----------------------GENERAL COVARIANCE--------------------------------
\section{Covariance matrix from parameter gradient}
\label{sec:cov}

We assume that there
is negligible spatial variation of each parameter 
inside the Hubble radius at the time of recombination,
so that the gradient can be 
considered a first-order perturbation.  We are therefore able to
calculate the anisotropy power spectra as a function of the cosmological
parameters, where we are considering that some parameters have a
dipolar variation over the sky, i.e.\ at last scattering parameter $X$ 
takes the form
\beq
X({\hat{\bf{n}}}) = X_0 + \Delta X\hat{\bf{n}}\cdot\hat{\bf{m}},
\eeq
where ${\hat{\bf{n}}}$ is the direction to the last scattering surface and 
${\hat{\bf{m}}}$ is the gradient direction.  The parameter takes the monopole 
or average value $X_0$ at the ``equator'', 
$\hat{\bf{n}}\cdot\hat{\bf{m}} = 0$, and the extreme values $X_0 \pm\Delta X$ 
at the ``poles'', ${\hat{\bf{n}}} = \pm{\hat{\bf{m}}}$.

%Let us consider how such a dipole would manifest itself on the CMB sky.
%For convenience, we represent the temperature anisotropies
%$\Theta({{\hat{\bf{n}}}})$ as an expansion in spherical harmonics:
%\beq
% \Theta({{\hat{\bf{n}}}})=\sum_{\ell=1}^{\infty}
%  \sum_{m=-\ell}^{\ell} a_{\ell m} Y_{\ell m }({{\hat{\bf{n}}}}) \,.
%\eeq
%For a small amplitude gradient in some parameter
%we expect the observed sky to consist of the sum of an isotropic sky 
%and a dipole modulated sky.
%
%The effect is similar in appearance to the boosting effects (or aberration) 
%due to our motion with respect to the CMB~\cite{Chluba:2011}.

   Generalizing the formalism developed in MSZB for a parameter gradient 
in an arbitrary direction, we can decompose the gradient into dipolar 
spherical harmonic components according to
\beq
\Delta X_m(\hat{\bf{m}}) \equiv \sqrt{\frac{4\pi}{3}}\Delta X\,Y_{1m}^*(\hat{\bf{m}}).
\eeq
In particular, we have
\begin{eqnarray}
\Delta X_0      &=& \Delta X\cos\theta_m\,,\\
\Delta X_{\pm1} &=& \mp\frac{\Delta X\sin\theta_m}{\sqrt2}e^{\mp i\phi_m}\,,
\end{eqnarray}
where $(\theta_m,\phi_m)$ are spherical coordinates for $\hat{\bf{m}}$.  
The modulated harmonic modes are then given by
\begin{eqnarray}
\nonumber a_{\ell m} = a_{\ell m}^{\rm iso}
 + \sqrt{\frac{4\pi}{3}}\sum_M\Delta X_M\sum_{\ell'}
   \sum_{m'=-\ell'}^{\ell'}\frac{da_{\ell'm'}^{\rm iso}}{dX}\\
\times \int d\Omega Y_{\ell m}^*(\hat{\bf{n}})
   Y_{\ell'm'}(\hat{\bf{n}})Y_{1,M}(\hat{\bf{n}})\, ,
\end{eqnarray}
where $a_{\ell m}^{\rm iso}$ are the statistically isotropic modes.  
The integral is evaluated using the Gaunt formula (see e.g.~\cite{Grad:1994}) 
to yield
\begin{eqnarray}
a_{\ell m} &=& a_{\ell m}^{\rm iso} + \sum_M\Delta X_M
   \sum_{\ell'm'}\frac{da_{\ell'm'}^{\rm iso}}{dX}\xi_{\ell m\ell'm'}^M\,.
\end{eqnarray}
The coupling coefficients $\xi^M_{\ell m\ell'm'}$ are given by
\begin{eqnarray}\label{eqn:xi0}
\xi_{\ell m \ell' m'}^0 &=& \delta_{m'm} \left[ \delta_{\ell' \, \ell-1}
 A_{\ell-1 \, m} + \delta_{\ell' \, \ell+1} A_{\ell  m}  \right]\,, \\
\label{eqn:xi1}
\xi_{\ell m \ell' m'}^{\pm1} &=& \delta_{m' \, m\mp1}
 \left[ \delta_{\ell' \, \ell-1} B_{\ell-1 \, \pm m-1} - \delta_{\ell' \, \ell+1}
 B_{\ell  \, \mp m}  \right]\,,
\end{eqnarray}
where
\begin{eqnarray}
A_{\ell m} &\equiv& \sqrt{\frac{\left( \ell + 1\right) ^2-m^2}
{\left( 2\ell+1\right)\left( 2\ell+3\right)}}\,,\\
\quad B_{\ell m} &\equiv&  \sqrt{\frac{\left(\ell+m+1 \right) \left(\ell+m +2 \right)}
{2\left( 2\ell+1\right)\left( 2\ell+3\right)}}\,.
\end{eqnarray}

From this we can compute the anisotropy covariance matrix to first order in 
$\Delta X$:
\begin{eqnarray}\label{eq:cov}
 C_{\ell m\ell^\prime m^\prime}
  &\equiv&\langle a_{\ell m}^\ast a_{\ell^\prime m^\prime} \rangle\nonumber\\
 &=& C_{\ell} \delta_{\ell \ell'} \delta_{m m'} + \frac{\delta C_{\ell \ell' }}{2}
 \sum_M \Delta X^*_M \xi_{\ell m \ell' m'}^M\,,
\end{eqnarray}
where
\beq
\delta C_{\ell\ell'} \equiv \frac{dC_{\ell}}{dX} + \frac{dC_{\ell'}}{dX}.
\eeq
Here $C_\ell \equiv \bra a_{\ell m}^{\rm iso*}a_{\ell m}^{\rm iso}\ket$ is the 
isotropic power spectrum.

%While the above covariance derivation
%is valid for arbitrary physical parameters, one has to be careful when
%considering parameters which produce isocurvature modes.
%Such gradients would produce an ``intrinsic'' non-negligible dipole in the CMB and 
%could include line of sight effects which have not been investigated here.
%Our method assumes that the gradient affects only the generation of {\em primary\/} 
%anisotropies (i.e.\ anisotropies generated at last scattering).
%We consider four parameters in our analysis: $\alpha$;
%the amplitude and spectral index of the primordial fluctuations,
%$A_{\rm s}$ and $n_{\rm s}$; as well as the primordial helium fraction
%$Y_{\rm p}$.  Unlike for the parameters of the standard cosmological model, it
%is unclear what the basic parameter set might be here.  Some researchers might
%prefer to stick with the usual cosmological parameters, while others might be
%more interested in particle physics quantities, such as
%$m_p/m_e$, other particle couplings, 
%mass ratios, and mixing angles, the effective 
%number of neutrino species, etc. 
%Whatever parameters are selected, one can extract
%power spectrum derivatives from standard CMB codes; we employed 
%\textsc{camb}~\cite{Lewis:1999bs}, specifically using the unlensed spectra.
%To simulate a change in
%$\alpha$, we adapted the \textsc{recfast} recombination code
%\cite{Seager:1999bc,Seager:1999km} as described in MSZB.
%Since gradients in different parameters produce very different derivative
%power spectra (see Fig.~2 of MSZB for examples), we could in principle
%differentiate between them. 







%-----------------GRADIENT ESTIMATOR------------------------------------------
\section{Gradient estimator}
\label{sec:grad est}
We would like to use the covariance matrix (\ref{eq:cov}) to determine best 
fit estimates for parameter gradients. We begin by maximizing the CMB 
likelihood over the parameter space $\{\Delta X_M\}$. With a Gaussian 
realization of temperatures, the CMB likelihood function is simply a 
Gaussian integral
\begin{eqnarray}
\mathcal{L}=\frac{1}{\sqrt{2\pi|C|}}e^{-d^\dagger C^{-1}d/2}\, ,
\end{eqnarray}
where $d$ is the vector of temperature anisotropy multipoles and $C$ is the 
covariance matrix.  We can equivalently maximize the log-likelihood function:
\begin{eqnarray}
- 2 \ln \mathcal{L} = \ln(2\pi) + \ln |C| + d^{\dagger} C^{-1} d\, ,
\end{eqnarray}
which is maximum for a particular $\Delta X_M$ when
\beq\label{eq:max}
d^{\dagger} C^{-1} \frac{d C}{d \Delta X_M^{}} C^{-1}  d
   = {\rm Tr} \left(C^{-1} \frac{d C}{d \Delta X_M^{}}  \right)\,.
\eeq
If we include a small anisotropy in the covariance matrix so that $C=C_I+C_A$, 
then we can expand $C_I^{-1}$ about the fiducial matrix $C_I$:
\beq
C^{-1} \approx C^{-1}_I - C_{I}^{-1} C_A C_I^{-1}\, .
\eeq
To first order in the anisotropy, Eq.~\eqref{eq:max} becomes
\begin{eqnarray}\label{eq:maxl}
&&d^{\dagger} C_I^{-1}\bigg( \frac{d C_A}{d \Delta X_M^{}} C_I^{-1}  d 
-  \frac{d C_A}{d\Delta X_M^{}} C_I^{-1}  C_A C_I^{-1}  d  - \\
&&C_A C_I^{-1} \frac{d C_A}{d \Delta X_M^{}} C_I^{-1}  d\bigg) 
+  {\rm Tr} \left( C_{I}^{-1} C_A C_I^{-1} \frac{d C_A}{d \Delta X_M^{}}\right)
 = 0\,    \nonumber
\end{eqnarray}
In terms of the covariance matrix in Eq.~\eqref{eq:cov}, we have, for 
example, that the $(\ell m),(\ell' m')$ entry of the matrix 
$C_I^{-1}dC_a/d\Delta X_M C_I^{-1}$ is given by
\beq
\frac{\delta C_{\ell'}\xi^M_{\ell m \ell 'm'}}{C_\ell C_{\ell'}}.
\eeq
Summing over all matrix entries $(\ell m),(\ell 'm')$, we can re-write the 
maximum likelihood condition, Eq.~\eqref{eq:maxl}, as
\begin{eqnarray}
&&\frac{\delta C_{\ell \ell'}}{C_{\ell} C_{\ell'}}\xi^i_{\ell m \ell' m'}
 a_{lm}^{\star} a_{\ell' m'}\nonumber \\
 &=&   \Delta X^{}_{M}  \frac{ \delta C_{\ell \ell'}^2 }{2 C_{\ell} C_{\ell'}} 
\bigg( \frac{a_{\ell m}^{\star} a_{\ell m}}{C_{\ell}} \xi^M_{\ell m \ell' m'} 
\xi^i_{\ell' m' \ell m} \nonumber \\
 &+&  \frac{a_{\ell m}^{\star} a_{\ell m}}{C_{\ell}} \xi^i_{\ell m \ell' m'} 
\xi^M_{\ell' m' \ell m} + \xi^M_{\ell m \ell' m'} \xi^i_{\ell' m' \ell m}\bigg)\, ,
\end{eqnarray}
where repeated indices are implicitly summed over. Since the 
$\xi^M_{\ell m \ell' m'}$ are orthogonal, we can solve this to obtain estimates 
for the $\Delta X_M$. Inputing the expressions for $\xi^M_{\ell m \ell' m'}$, 
Eq.~\eqref{eqn:xi0}, and using the symmetry of the covariance matrix, we get
\begin{eqnarray}\label{eqn:delXest}
\Delta X_0^{} &=&  \frac{2 \sum_{\ell m} 
\frac{\delta C_{\ell \ell+1}}{ C_{\ell}C_{\ell+1}} A_{\ell m}
  a _{\ell m}^{\star} a_{\ell+1 m}  }{\sum_{\ell m}
  \frac{\delta C_{\ell \ell+1}^2}{ C_{\ell}C_{\ell+1}} A_{\ell m}^2
 \left(  \frac{a^{\star}_{\ell m}  a_{\ell m}}{C_{\ell}}
 + \frac{a^{\star}_{\ell+1 \, m}  a_{\ell+1 \, m}}{C_{\ell+1}}  -1  \right)}
 \,,\\
\Delta X_{+1} &=&  \frac{2 \sum_{\ell m} 
\frac{\delta C_{\ell \ell+1}}{ C_{\ell}C_{\ell+1}} B_{\ell m}
  a _{\ell m}^{\star} a_{\ell+1 \, m+1}  }{\sum_{\ell m}
  \frac{\delta C_{\ell \ell+1}^2}{ C_{\ell}C_{\ell+1}} B_{\ell m}^2
 \left( \frac{a^{\star}_{\ell m}  a_{\ell m}}{C_{\ell}}
  + \frac{a^{\star}_{\ell+1 \, m+1}  a_{\ell+1 \, m+1}}{C_{\ell+1}}  -1  \right)} \,,
\end{eqnarray}
and $\Delta X_{-1}=-\Delta X_{+1}^\star$.
The best fit gradient parameters are then just
\begin{eqnarray}\label{eqn:estimates1}
\Delta X &=& \sqrt{\Delta X_0^2 + 2|\Delta X_1|^2 }\,,\\
\label{eqn:estimates2}\theta_m &=& \cos^{-1}\left( \Delta X_0/\Delta X\right)\,,\\
\label{eqn:estimates3}\phi_m &=& \tan^{-1} \left[ - {\rm Im}
 \left( \Delta X_{+1} \right) / {\rm Re}  \left(\Delta X_{+1} \right)   \right]\,.
\end{eqnarray}
In Sec.~\ref{sec:results} we compute these estimates for a  {\em WMAP\/} 
coadded nine year sky map. For the remainder of the paper, we fix the 
parameter monopoles $X_0$ to their respective {\it WMAP\/}-9 best-fit 
values~\cite{Komatsu:2010fb}. 

%We note that inclusion of polarization
%improves sensitivity by a factor of about two for the aberration
%effect~\cite{Amendola:2010ty}, and we would expect a similar improvement here.


%For our purposes, since we expect
%$\Delta X/X_{0}$ to be small, we can use $\Delta X/X_{0}=0$ as the fiducial
%point at which to evaluate the covariance inverse and derivatives.  The
%derivative matrix is determined using finite differences,
%and is given explicitly by
%\beq \label{eqn:deriv}
% \ld(C_{,\Delta X_{i}}\rd)_{\ell m\ell^\prime m^\prime}=\frac{X_{i0}}{2}
%  \left( \frac{dC_{\ell}}{dX_{i}} + \frac{dC_{\ell^\prime}}{dX_{i}} \right)
%  \xi_{\ell m \ell^\prime m^\prime}^{0}\,.
%\eeq
%Here $C^{-1}$ is diagonal due to our choice of fiducial point.
%The product
%$C^{-1}C_{,\Delta X_{j}}C^{-1}$ can then be simplified so that any entry $p,q$
%(row $p$, column $q$) is given by
%\beq\label{eqn:product1}
% \ld(C^{-1}C_{,\Delta X_{j}}C^{-1}\rd)_{pq}
%  = C_{pp}^{-1}\ld(C_{,\Delta X_{j}}\rd)_{pq}C_{qq}^{-1},
%\eeq
%Pre-multiplying this result by $C_{,\Delta X_{i}}$ yields the final matrix used
%in the Fisher calculation.  Since the Fisher matrix involves only the trace of
%this product, we need only compute its diagonal entries.

%We perform the analysis for an ideal cosmic variance-limited experiment,
%as well as for a more realistic case including experimental noise, with
%experimental
%specifications from {\em Planck}~\cite{Efstathiou:1999,PlanckI}.
%This is done by adding a diagonal noise
%covariance matrix to the above covariance matrices with entries given by
%(e.g.~\cite{Foreman:2011uj})
%\beq\label{eqn:noise}
% N_{\ell}=\left[\sum_{c}\frac{1}{\theta_{c}(\Delta_{c})^{2}}
%  \exp\left(-\frac{\ell^{2}\theta_{c}^{2}}{8\ln2}\right)\right]^{-1}.
%\eeq
%Here, we take the sum over $c$ to be a sum over the four highest frequency
%channels of {\it Planck}, $\theta_{c}$ is the FWHM beam size, and
%$\Delta_{c}$ is the noise per pixel.
%The Fisher matrices are evaluated out to $\ell=2000$, and we find that the
%values for {\it Planck\/} are almost indistinguishable from those obtained
%with an experiment which is cosmic variance limited to that multipole.
%By considering the diagonal entries of the
%inverse Fisher matrix, we can estimate errors on each parameter gradient using
%the Cramer-Rao inequality
%\beq
% \sigma (\Delta X_{i}/X_{i})\ge\sqrt{(F^{-1})_{ii}}.
%\eeq
%The results are given in Table~\ref{table:errors}.  We see that the gradient in
%$\alpha$ is more tightly constrained than the other parameters.  This is the
%result of its unique associated derivative power spectrum (see MSZB).

%\begin{table}
%\caption{Forecasted errors $\sigma(\Delta X/X)\times10^{-3}$ for
%$\ell_{\rm max}=2000$.  These are for the parameters of the {\it Planck\/}
%experiment, although we note that the values we found for a cosmic variance
%limited experiment only differ in the third or fourth digit.
%The lower part of the table corresponds to the case where two co-aligned
%gradients are present on the same sky.}
%\label {table:errors}
%\nointerlineskip
%\setbox\tablebox=\vbox{
%%
%\newdimen\digitwidth % These five lines change what an asterisk
%\setbox0=\hbox{\rm 0} % means to TeX. Instead of meaning
%\digitwidth=\wd0 % "print an '*' here", it now means "leave
%\catcode`*=\active % as much blank space as a single number
%\def*{\kern\digitwidth} % takes up".
%%
%\newdimen\signwidth % These five lines change the meaning of an
%\setbox0=\hbox{{\rm +}} % exclamation mark in the same way, so that it
%\signwidth=\wd0 % leaves as much space as a plus or minus sign.
%\catcode`!=\active % These definitions will disappear at the end of
%\def!{\kern\signwidth} % the \vbox.
%%
%\newdimen\pointwidth % These five lines change the meaning of a
%\setbox0=\hbox{\rm .} % question mark in the same way, so that it
%\pointwidth=\wd0 % leaves as much space as a period.
%\catcode`?=\active % These definitions will disappear at the end of
%\def?{\kern\pointwidth} % the \vbox.
%%
%%\halign{\hbox to 1.75in{#\leaderfil}\tabskip=2.0em&
%\halign{\hbox to 1.75in{#}\tabskip=2.0em&
%\hfil#\hfil\tabskip=1.0em\cr
%\noalign{\doubleline}
%\omit  Parameter &$\sigma\times10^3$ \cr
%\noalign{\vskip 3pt\hrule\vskip 4pt}
%$A_{\rm s}$&           *3.7* \cr
%$n_{\rm s}$&           *6.8* \cr
%$\alpha$&              *0.25 \cr
%$Y_{\rm p}$&           35?** \cr
%\noalign{\doubleline}%{\vskip 3pt\hrule\vskip 4pt}
%}}
%\endtablenow
%\end{table}
%\begin{table}
%\caption{Forecasted errors $\sigma(\Delta X/X)\times10^3$ for
%$\ell_{\rm max}=2000$.  These are for the parameters of the {\it Planck\/}
%experiment, although we note that the values we found for a cosmic variance
%limited experiment only differ in the third or fourth digit.
%\label{table:errors}}
%\begin{ruledtabular}
%\begin{tabular}{ccdc}
%&Parameter $X$&\multicolumn{1}{c}{$\sigma\times10^3$}&\\\hline
%&$A_{\rm s}$  &           3.7 &\\
%&$n_{\rm s}$  &           6.8 &\\
%&$\alpha$     &           0.25&\\
%&$Y_{\rm p}$  &          35   &
%\end{tabular}
%\end{ruledtabular}
%\end{table}

%To illustrate the correlation between different gradients, we produce
%confidence ellipses, in two dimensional sub-spaces,
%marginalizing over the other parameters, as shown in Fig.~\ref{fig:ellipses}.
%We see that for an experiment reaching $\ell=2000$, degeneracies between parameters
%are small enough to allow us to distinguish between them at the $10^{-3}$ level.
%
%This analysis addresses the case where all four of these gradients co-exist in 
%the same direction, which is presumed to be known. Note, however, that the errors 
%given in Table~\ref{table:errors} are completely independent of the actual value 
%of any such gradients (so long as the gradients are small enough that our 
%fiducial model of $\Delta X=0$ is valid). In other words, these constraints are 
%equally applicable if any subset of these four gradients are present, so long 
%as they lie in the same direction. 

%We might also like to consider the situation 
%in which different parameters have tilts in different directions.  To simplify 
%the problem, we consider the case with exactly two different gradients separated
%by an angle $\theta$. To compute the off-diagonal elements of the new Fisher 
%matrix, we orient one of the gradients along the polar axis. This will have 
%exactly the same covariance derivative matrix as previously calculated. For the 
%other parameter, $\Delta X_{j}$, we must include the full derivative obtained from 
%the covariance matrix.
%%\begin {eqnarray}
%%(C_{,\Delta X_{j}})_{\ell m \ell^\prime m^\prime}&=&\frac{X_{j0}}{2}
% % \left[ \frac{dC_{\ell}}{dX_{j}} + \frac{dC_{\ell^\prime}}{dX_{j}} \right]\\
%% &\times& \left[\xi^{0}_{\ell m\ell^\prime m^\prime}\cos\theta
%%+\frac{\sin\theta}{\sqrt{2}}\bigg(\frac{\xi^{-1}_{\ell m\ell^\prime m^\prime}}
%%{e^{i\phi}}-\frac{\xi^{+1}_{\ell m \ell^\prime m^%\prime}}{e^{-i\phi}}\bigg)\right]
%%\nonumber\,\,\,\,
%Using Eq.~(\ref{eqn:fisher}) one can show that the resultant Fisher matrix 
%for such a case depends only on the separation angle $\theta$,
%and not on which parameter we choose to orient along the pole.
%Now using Eq.~(\ref{eqn:product1}) and the symmetry of the covariance 
%derivatives, we find that the $m^\prime=m-1$ and $m^\prime=m+1$ off-diagonal 
%terms do not affect the Fisher matrix.  For the diagonal Fisher components, it 
%is clear that the computation is identical to what was done originally. Thus 
%for any two \emph{different} gradients separated by an angle $\theta$,
%their resultant Fisher matrix $F^\prime_{ij}$ can be computed as
%\beq
%F^\prime_{ij}=kF_{ij},
%\eeq
%where
%\beq
%   k = \left\{
%     \begin{array}{ll}
%       1                & i=j,\\
%       \cos\theta \quad & i\neq j,
%     \end{array}
%   \right.
%\eeq
%and $F_{ij}$ is our original Fisher matrix marginalized over all parameters 
%except $X_{i}$ and $X_{j}$.

   To obtain errors one can expand the log-likelihood as a quadratic function 
around the best-fit.  The Fisher matrix is defined as
\beq\label{eqn:fisher}
F_{MM'} \equiv\frac{1}{2}{\rm Tr}\ld(C_{,\Delta X_M}C^{-1}C_{,\Delta X_{M'}}C^{-1}\rd).
\eeq
Here $C_{,\Delta X_M}$ denotes the derivative of the covariance matrix with
respect to the gradient component $\Delta X_M$.  For small $\Delta X$ this is 
given explicitly by
\beq
F_{MM'} \simeq \frac{\delta C_{\ell \ell'}^2}
{8 C_{\ell} C_{\ell'}}  \xi^M_{\ell m \ell' m'} \xi^{M'}_{\ell' m' \ell m}\,,
\eeq
which, after transforming the basis, gives the non-zero terms
\begin{eqnarray}
F_{0,0} &=& \frac{1}{4}\sum_{\ell m}
 \frac{\delta C_{\ell \ell+1}^2}{ C_{\ell} C_{\ell+l}}  A_{\ell m}^2 \,, \\ \nonumber 
F_{ {\rm Re}(\Delta X_{+1}^{})  {\rm Re}(\Delta X_{+1}^{}) }
 &=&  \frac{1}{2}\sum_{\ell m} \frac{\delta C_{\ell \ell+1}^2}{ C_{\ell} C_{\ell+l}}
  B_{\ell m}^2\, ,
\end{eqnarray}
with $F_{{\rm Im}(\Delta X_{+1}^{})  {\rm Im}(\Delta X_{+1}^{}) } 
= F_{ {\rm Re}(\Delta X_{+1}^{})  {\rm Re}(\Delta X_{+1}^{}) }$.  
Thus from the Cramer-Rao inequality, we can estimate the standard errors as
\begin{eqnarray} \label{eqn:errorest}
\sigma_{\Delta X_0^{}} &=& \frac{2}{\sqrt{\sum_{\ell m}
   \frac{\delta C_{\ell\ell+1}^2}{C_{\ell}C_{\ell+l}} A_{\ell m}^2}}\,,\\\nonumber 
\sigma_{ {\rm Re}(\Delta X_{+1}^{})  } = \sigma_{ {\rm Im}(\Delta X_{+1}^{})}
    &=&  \sqrt{\frac{2}{\sum_{\ell m}
         \frac{\delta C_{\ell \ell+1}^2}{ C_{\ell} C_{\ell+l}}  B_{\ell m}^2}}\,.
\end{eqnarray}








%----------------Results-----------------------
\section{Results}
\label{sec:results}







%Estimates of the amplitude and direction of the dipole modulation of power
%were obtained in section~\ref{sec:grad est}.
%This is {\it similar\/} to the published searches for modulation of
%$A_{\rm s}$, except that we would find a signal even if some parameter other
%than $A_{\rm s}$ was being modulated.

%We determine the parameter $\Delta A_{\rm s}/A_{\rm s}$ for the {\it WMAP\/}-9 data. 
%Since the Galactic foreground has a strong effect on the signal, we remove 
%it by applying a $20^\circ$ symmetric cut to the V band map. Ideally, we 
%would use a temperature mask and an inverse variance filtered map to remove 
%any multipole coupling induced by the mask. Such a procedure can be found 
%in Ref.~\cite{Hanson:2009gu}, but it is left for future work to implement 
%this here. Instead, we test the estimators on random maps with a simulated 
%gradient in the scalar amplitude, and with symmetric Galactic cuts of 
%$20^\circ$. We find the cut induces only a small bias with respect to 
%the Fisher errors derived above. 
%%In fact, a reasonable scaling for the uncertainty of the estimators due to 
%%the cut is simply $\sigma_{\rm fisher}/\sqrt{f_{\rm vis}}$
% %where $f_{\rm vis}$ is the fraction of the sky unmasked.
%The lack of strong coupling from the mask is not unexpected, since by 
%construction, it has no dipole component.

%Figure \ref{fig:wmap} shows each of the $\Delta X_i/X$ component estimates 
%for the {\it WMAP\/}-9 data, where $X=A_s$. Combining these results for 
%$\ell_{\rm max}=200$ and propagating the errors mentioned above through 
%equations \eqref{eqn:estimates1}, \eqref{eqn:estimates2}, and 
%\eqref{eqn:estimates3}, we arrive at the values:
%\begin{eqnarray}
%\Delta A_s/A_s &=& 0.076\pm0.014\, ; \\
%\theta&=& 108^{\circ}\pm7^{\circ}\, ;\\
%\phi&=& 73^{\circ}\pm 5^{\circ}\, .
%\end{eqnarray}
%We note that inclusion of a noise covariance matrix of the form 
%Eq.~\ref{eqn:noise} with {\em WMAP} experimental parameters has a negligible 
%effect on the errors listed above. However, while the Galactic cut does not 
%appreciably bias the signal, it does increase the variance in the noise, 
%suggesting that the above uncertainties are somewhat underestimated.

%Note that the apparent dipole in
%$A_{\rm s}$ could be a gradient in many combinations of
%parameters, as they are highly degenerate at low $\ell_{\rm max}$.
%On a speculative note, we might also mention that the dependence of the
%coupling on $\ell$ has roughly the character expected for a gradient in the
%tensor amplitude $A_{\rm t}$, since the primordial gravity wave contribution
%to the CMB anisotropies falls off after $\ell\simeq50$.  However, a more sensitive
%measurement of the dipole modulation would be required in order to decide
%what parameter combination fits best.







%----------------- CONCLUSIONS -----------------------
\section{Discussion}
\note{To be completely rewritten}  

We have shown that, despite partial degeneracies that exist between parameter
gradients, it should be possible to determine the values of $\Delta X/X$
causing a dipole modulation, provided that the amplitude is larger than
about the $10^{-3}$ level. We note that although extending this Fisher
analysis to any other parameters is straightforward, we cannot predict the 
effect this would have on the constraints listed above. It is possible that a 
new parameter could be partially degenerate with
one of the parameters considered here, for example.

If a gradient does exist at all, then it is possible that there could be
more than one parameter with a gradient.  These could either be aligned, or
in different directions, and we have shown how to deal with both cases using
the formalism of the $\xi^M$ matrices.

In practice the limitation for measuring gradients
is likely to be systematic effects, rather than
instrumental noise.  For small $\Delta X$, it 
may be difficult to distinguish the correlations from those due to 
the doppler boosting effect.  However, weaker gradients may be detectable 
in the future using the extra information contained in 21-cm surveys 
\cite{Lewis:2007kz}.  Still, it will be necessary in future, with more sensitive
searches for gradients, to fully account for the correlations due to
foregrounds and masking, as well as other systematics.

%[I THINK WE SHOULD LOSE THIS PARAGRAPH]
%In addition, for some parameters 
%(such as $\alpha$), it is possible that a gradient may generate a 
%temperature dipole through the effect of the parameter on epochs prior to
%recombination (e.g.\ reheating).  
%Although the details would be model dependent, we 
%would expect that the generated temperature dipole would be of order 
%$\Delta X/X_0$, and hence, for the {\it Planck\/} limit, of $\sigma (\Delta 
%X/X_0) \approx 10^{-3}$, 
%will likely be negligible in comparison with the observed 
%(locally generated) temperature dipole.  In other words, the effects 
%of the gradient {\em local} to the LSS are expected to be larger 
%than any effects at very early times.

Spatial variation of fundamental constants has
been considered before (e.g.~\cite{BarrowOToole01,Donoghue03,Sigurdson03}),
but in MSZB we showed how a gradient across our Hubble patch generically
gives rise to correlations between neighbouring multipoles in the CMB 
anisotropy spectrum.  
In this paper we have used the same approach to investigate current data from
{\it WMAP}.  We have also extended the analysis to show how one can consider
degeneracies among parameters, even if the gradients are in different
directions.

There is of course no reason to {\it expect\/} such gradients to exist at
measurable levels.  Nevertheless, the search for gradients is straightforward
and requires the same data which are required for more conventional
studies of cosmological parameters.  {\it If\/} a convincing gradient was
found it would be extremely important as an indicator of exotic early Universe
physics.




%----------------- ACKNOWLEDGMENTS -----------------------
\section*{Acknowledgments} 
%{\em Acknowledgments.---}% 
This research was supported by the Natural Sciences and Engineering 
Research Council of Canada and the Canadian Space Agency.


\bibliography{acrossU}

\end{document}

